%
%
%  This file is inserted in the file Segmentation.tex
%
%

\subsection{Overview}
\label{sec:AboutWatersheds}
\index{Watersheds}
\index{Watersheds!Overview}
Watershed segmentation classifies pixels into regions using gradient descent on
image features and analysis of weak points along region boundaries.  Imagine
water raining onto a landscape topology and flowing with gravity to collect in
low basins.  The size of those basins will grow with increasing amounts of
precipitation until they spill into one another, causing small basins to merge
together into larger basins.  Regions (catchment basins) are formed by using
local geometric structure to associate points in the image domain with local
extrema in some feature measurement such as curvature or gradient magnitude.
This technique is less sensitive to user-defined thresholds than classic
region-growing methods, and may be better suited for fusing different types of
features from different data sets.  The watersheds technique is also more
flexible in that it does not produce a single image segmentation, but rather a
hierarchy of segmentations from which a single region or set of regions can be
extracted a-priori, using a threshold, or interactively, with the help of a
graphical user interface
\cite{Yoo1992,Yoo1991}.

The strategy of watershed segmentation is to treat an image $f$ as a height
function, i.e.,  the surface formed by graphing $f$ as a function of its
independent parameters, $\vec{x} \in U$.  The image $f$ is often not the
original input data, but is derived from that data through some filtering,
graded (or fuzzy) feature extraction, or fusion of feature maps from different
sources.  The assumption is that higher values of $f$ (or $-f$) indicate the
presence of boundaries in the original data.  Watersheds may therefore be
considered as a final or intermediate step in a hybrid segmentation method,
where the initial segmentation is the generation of the edge feature map.

Gradient descent associates regions with local minima of $f$ (clearly interior
points) using the watersheds of the graph of $f$, as in
Figure~\ref{fig:segment}.
\begin{figure}
\centering
\includegraphics[width=0.9\textwidth]{WatershedCatchmentBasins.eps}
\itkcaption[Watershed Catchment Basins]{A fuzzy-valued boundary map, from an image
  or set of images, is segmented using local minima and catchment basins.}
\protect\label{fig:segment}
\end{figure}
That is, a segment consists of all points in $U$ whose paths of steepest
descent on the graph of $f$ terminate at the same minimum in $f$.  Thus, there
are as many segments in an image as there are minima in $f$.  The segment
boundaries are ``ridges'' \cite{Koenderink1979,Koenderink1993,Eberly1996} in
the graph of $f$.  In the 1D case ($U \subset \Re$), the watershed boundaries
are the local maxima of $f$, and the results of the watershed segmentation is
trivial.  For higher-dimensional image domains, the watershed boundaries are
not simply local phenomena; they depend on the shape of the entire watershed.

The drawback of watershed segmentation is that it produces a region for each
local minimum---in practice too many regions---and an over segmentation
results.  To alleviate this, we can establish a minimum watershed depth.  The
watershed depth is the difference in height between the watershed minimum and
the lowest boundary point.  In other words, it is the maximum depth of water
a region could hold without flowing into any of its neighbors.  Thus, a
watershed segmentation algorithm can sequentially combine watersheds whose
depths fall below the minimum until all of the watersheds are of sufficient
depth.  This depth measurement can be combined with other saliency
measurements, such as size.  The result is a segmentation containing regions
whose boundaries and size are significant.  Because the merging process is
sequential, it produces a hierarchy of regions, as shown in
Figure~\ref{fig:watersheds}.
\begin{figure}
\centering
\includegraphics[width=0.9\textwidth]{WatershedsHierarchy.eps}
\itkcaption[Watersheds Hierarchy of Regions]{A watershed segmentation combined
with a saliency measure (watershed depth) produces a hierarchy of regions.
Structures can be derived from images by either thresholding the saliency
measure or combining subtrees within the hierarchy.}
\protect\label{fig:watersheds}
\end{figure}
Previous work has shown the benefit of a user-assisted approach that provides
a graphical interface to this hierarchy, so that a technician can quickly move
from the small regions that lie within an area of interest to the union of
regions that correspond to the anatomical structure \cite{Yoo1991}.

There are two different algorithms commonly used to implement watersheds:
top-down and bottom-up.  The top-down, gradient descent strategy was chosen for
ITK because we want to consider the output of multi-scale differential
operators, and the $f$ in question will therefore have floating point
values. The bottom-up strategy starts with seeds at the local minima in the
image and grows regions outward and upward at discrete intensity levels
(equivalent to a sequence of morphological operations and sometimes called {\em
morphological watersheds} \cite{Serra1982}.) This limits the accuracy by
enforcing a set of discrete gray levels on the image.

\begin{figure}
\centering
\includegraphics[width=0.9\textwidth]{WatershedImageFilter.eps}
\itkcaption[Watersheds filter composition]{The construction
of the Insight watersheds filter.}
\protect\label{fig:constructionWatersheds}
\end{figure}

Figure~\ref{fig:constructionWatersheds} shows how the ITK image-to-image
watersheds filter is constructed.  The filter is actually a collection of
smaller filters that modularize the several steps of the algorithm in a
mini-pipeline.  The segmenter object creates the initial segmentation via
steepest descent from each pixel to local minima. Shallow background regions
are removed (flattened) before segmentation using a simple minimum value
threshold (this helps to minimize oversegmentation of the image).  The
initial segmentation is passed to a second sub-filter that generates a
hierarchy of basins to a user-specified maximum watershed depth.  The
relabeler object at the end of the mini-pipeline uses the hierarchy and the
initial segmentation to produce an output image at any scale {\em below} the
user-specified maximum.  Data objects are cached in the mini-pipeline so that
changing watershed depths only requires a (fast) relabeling of the basic
segmentation.  The three parameters that control the filter are shown in
Figure~\ref{fig:constructionWatersheds} connected to their relevant
processing stages.

\subsection{Using the ITK Watershed Filter}
\label{sec:UsingWatersheds}
\index{Watersheds!ImageFilter}
\input{WatershedSegmentation.tex}

%\subsection{Interpreting the Results}
%\label{sec:VisualizingWatersheds}
%\index{Watersheds!Visualization}
%In order to interpret the output of the Insight watersheds algorithm, it is
%important to understand what the output represents and how it is formatted. The
%itk::WatershedImageFilter produces an image of unsigned long integers.  Each
%integer number is a label for a unique segmented region (catchment basin) from
%the original input.  The output is the same size and dimensionality of the
%input.

%Because the segmented image may have potentially many thousands of labels, some
%care must be taken when visualizing the data or information may be lost.  One
%effective way to visualize the output is to map the integer labels into
%distinct RGB colors.  Because labels close in value tend to also be close
%spatially in the image, it is helpful to spread sequential label values far
%apart in the RGB range.  A hashing scheme that puts more weight on the
%least-significant integer bits is a good way to accomplish this.
%Figure~\ref{fig:colorVisWatersheds} shows a slice taken from a segmentation of
%a section of abdomen from the Visible Female Cryosection data.  The unsigned
%long label values of the output have been hashed into RGB colors.

%\begin{figure}
%\centering
%\includegraphics[width=.95\textwidth]{WatershedAbdomenSegmentation.eps}
%\itkcaption[Watershed segmentation of visible woman data]{A slice from a
%segmentation of Visible Female cryosection data.
%The original is shown at the left and the segmented image is shown to the
%right. Colored regions in the segmented image correspond to structures in the
%original data. }
%\protect\label{fig:colorVisWatersheds}
%\end{figure}

%For volumetric data, it is often interesting to create a surface rendering of
%one or more regions in the output.  This can be done by thresholding the
%region(s) of interest from the output image and exporting the result to a
%visualization package capable of isosurface rendering.  Thresholding can be
%done either by explicit manipulation of the image values through an ITK image
%iterator, or using one of the several Insight image thresholding filters.

%Figure~\ref{fig:surfaceRenderingWatersheds} is a surface rendering of the right
%eye, the optic nerve and chiasm, the lens of the eye, and the right lateral
%rectus muscle.  A slice from the original Visible Female head and neck
%cryosection data from which the segmentations were created is shown at the
%left.  This image was created as described above by thresholding isovalues in a
%watershed segmentation output and then rendered using third-party visualization
%software.

%\begin{figure}
%\centering
%\includegraphics[width=0.95\textwidth]{WatershedRendering.eps}
%\itkcaption[Watershed segmenation visualization]{A surface rendering (right)
%of four anatomical structures in the Visible
%Female head and neck. A slice of the data from which the segmentation was
%created is shown at the left.  The right and left optic nerves and chiasm are
%shown in yellow.  The right eye is in transparent purple.  The lens is dark
%purple.  The structure in red is the lateral rectus muscle.}
%\protect\label{fig:surfaceRenderingWatersheds}
%\end{figure}

