% Please do NOT edit this file.
% It has been automatically generated
% by a perl script from the original cxx sources
% in the Insight/Examples directory

% Any changes should be made in the file
% /ORFEO/thomas/ORFEO-TOOLBOX/OTB/Examples/StartExamples/MeanImageFilter.cxx


Le code source se trouve dans le fichier\\
\texttt{Examples/StartExamples/MeanImageFilter.cxx}.


  BLA BLA...

  \begin{center}
  \begin{picture}(200,46)
  \put(   5.0,  0.0 ){\framebox(30.0,15.0){25}} 
  \put(  35.0,  0.0 ){\framebox(30.0,15.0){30}} 
  \put(  65.0,  0.0 ){\framebox(30.0,15.0){32}} 
  \put(   5.0, 15.0 ){\framebox(30.0,15.0){27}} 
  \put(  35.0, 15.0 ){\framebox(30.0,15.0){25}} 
  \put(  65.0, 15.0 ){\framebox(30.0,15.0){29}} 
  \put(   5.0, 30.0 ){\framebox(30.0,15.0){28}} 
  \put(  35.0, 30.0 ){\framebox(30.0,15.0){26}} 
  \put(  65.0, 30.0 ){\framebox(30.0,15.0){50}} 
  \put( 100.0, 22.0 ){\vector(1,0){20.0}}
  \put( 125.0, 15.0 ){\framebox(34.0,15.0){30.22}} 
  \put( 160.0, 22.0 ){\vector(1,0){20.0}}
  \put( 185.0, 15.0 ){\framebox(30.0,15.0){30}} 
  \end{picture}
  \end{center}

  Suite BLA BLA...

  \index{itk::MeanImageFilter}


  The header file corresponding to this filter should be included first.

  \index{itk::MeanImageFilter!header}

\small
\begin{verbatim}
#include "itkMeanImageFilter.h"
\end{verbatim}
\normalsize
  
    Then the pixel types for input and output image must be defined and, with
    them, the image types can be instantiated.
  
\small
\begin{verbatim}
  typedef   unsigned char  InputPixelType;
  typedef   unsigned char  OutputPixelType;

  typedef itk::Image< InputPixelType,  2 >   InputImageType;
  typedef itk::Image< OutputPixelType, 2 >   OutputImageType;
\end{verbatim}
\normalsize
  
    Using the image types it is now possible to instantiate the filter type
    and create the filter object. 
  
    \index{itk::MeanImageFilter!instantiation}
    \index{itk::MeanImageFilter!New()}
    \index{itk::MeanImageFilter!Pointer}
   
\small
\begin{verbatim}
  typedef itk::MeanImageFilter<
               InputImageType, OutputImageType >  FilterType;

  FilterType::Pointer filter = FilterType::New();
\end{verbatim}
\normalsize
  
    The size of the neighborhood is defined along every dimension by
    passing a \code{SizeType} object with the corresponding values. The
    value on each dimension is used as the semi-size of a rectangular
    box. For example, in $2D$ a size of \(1,2\) will result in a $3 \times
    5$ neighborhood.
  
    \index{itk::MeanImageFilter!Radius}
    \index{itk::MeanImageFilter!Neighborhood}
  
\small
\begin{verbatim}
  InputImageType::SizeType indexRadius;
  
  indexRadius[0] = atoi(argv[3]); // radius along x
  indexRadius[1] = atoi(argv[4]); // radius along y

  filter->SetRadius( indexRadius );
\end{verbatim}
\normalsize
  
    The input to the filter can be taken from any other filter, for example
    a reader. The output can be passed down the pipeline to other filters,
    for example, a writer. An update call on any downstream filter will
    trigger the execution of the mean filter.
  
    \index{itk::MeanImageFilter!SetInput()}
    \index{itk::MeanImageFilter!GetOutput()}
  
\small
\begin{verbatim}
  filter->SetInput( reader->GetOutput() );
  writer->SetInput( filter->GetOutput() );
  writer->Update();
\end{verbatim}
\normalsize
   
   \begin{figure}
   \center
   \includegraphics[width=0.44\textwidth]{Circle.eps}
   \includegraphics[width=0.44\textwidth]{CircleMeanOutput.eps}
   \itkcaption[Effect of the MedianImageFilter]{Effect of the MeanImageFilter on point.}
   \label{fig:CircleMeanOutput}
   \end{figure}
  
    Figure \ref{fig:CircleMeanOutput} illustrates the effect of this
    filter on an image of a point avec voisinage de \(10,10\) correspondant a un filtre de taille $ 21 \times 21 $.
  
