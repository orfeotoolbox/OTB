\chapter{Basic Filtering}

Ce chapitre pr\'{e}sente des exemples de filtrages ........


\section{Speckle reduction filters}
\label{sec:SpeckleFilters}
\ifitkFullVersion
\input{LeeImageFilter.tex}
\fi


\section{Filtrage de voisinage}
\label{sec:NeighborhoodFilters}

BLA BLA....


\subsection{Filtre Moyen}
\label{sec:MeanFilter}

\ifitkFullVersion
#Add a dependency of "CircleMeanOutput.png" on "Circle.png".
ADD_GENERATED_FIG_DEPS( "CircleMeanOutput.png" "Circle.png" )
# Cmake macro to invoke: /ORFEO/thomas/ORFEO-TOOLBOX/otb/OTB/Examples/Data/MeanImageFilter  /Circle.png /CircleMeanOutput.png 10 10
RUN_EXAMPLE( "MeanImageFilter" "CircleMeanOutput.png" "/ORFEO/thomas/ORFEO-TOOLBOX/otb/OTB/Examples/StartExamples/MeanImageFilter.cxx"  /Circle.png /CircleMeanOutput.png 10 10 )
CONVERT_IMG( "Circle.png" "Circle.eps" "" )
ADD_DEP_TEX_ON_EPS_FIGS( "/ORFEO/thomas/ORFEO-TOOLBOX/otb/OTB-Documents/SoftwareGuide/Art/Generated" "Circle.eps" )
CONVERT_IMG( "CircleMeanOutput.png" "CircleMeanOutput.eps" "" )
ADD_DEP_TEX_ON_EPS_FIGS( "/ORFEO/thomas/ORFEO-TOOLBOX/otb/OTB-Documents/SoftwareGuide/Art/Generated" "CircleMeanOutput.eps" )

\fi

BLA BLA Mean filter


The typical effect of median filtration on a noisy digital image is a dramatic reduction in impulse noise spikes. The filter also tends to preserve brightness differences across signal steps, resulting in reduced blurring of regional boundaries. The filter also tends to preserve the positions of boundaries in an image.

Figure \ref{fig:BinaryMedianImageFilterOutputMultipleIterations} below shows the effect of running the median filter with a 3x3 classical window size 
1, 10 and 50 times. There is a tradeoff in noise reduction and the sharpness of the image when the window size is increased\begin{figure}
  \center
  \includegraphics[width=0.50\textwidth]{itkLogo.eps}
  \itkcaption[Caption]{Image de tests : temporaire !!!!}
  \label{fig:itkLogo}
\end{figure}.



