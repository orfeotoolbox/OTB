\setcounter{secnumdepth}{3}

\chapter{Compiling OTB from source}
\label{chapter:Installation}
\index{Installation}

There are two ways to install OTB library on your system: installing from a binary distribution or compiling from sources. 
You can find information about the installation of binary packages for OTB-Applications and Monteverdi and Monteverdi2 (if they are available on your platform) in the OTB-Cookbook.
This chapter covers installation from sources, also known as compiling.

OTB has been developed and tested across different combinations of operating systems, compilers, and hardware platforms including MS-Windows, Linux on Intel-compatible hardware and Mac OSX.  It is known to work with the following compilers in 32/64 bit:
\begin{itemize}
\item Visual Studio 2010 and higher compiler on MS-Windows
\item GCC 4.1 and higher on Unix/Linux systems
\item Clang on MacOSX (10.8 and higher) systems
\end{itemize}

\index{CMake}
The challenge of supporting OTB across platforms has been solved through the use of CMake, a cross-platform, open-source
build system. CMake is used to control the software compilation process using simple platform and compiler independent
configuration files.  CMake generates native makefiles and workspaces that can be used in the compiler environment of
your choice. CMake is quite sophisticated: it supports complex environments requiring system configuration, compiler
feature testing, and code generation.

CMake generates Makefiles under UNIX systems and generates Visual Studio workspaces under Windows (and appropriate build
files for other compilers like Borland). The information used by CMake is provided by \code{CMakeLists.txt} files that
are present in every directory of the OTB source tree. These files contain information that the user provides to CMake
at configuration time. Typical information includes paths to utilities in the system and the selection of software
options specified by the user.

CMake runs in an interactive mode in that you iteratively select
options and configure according to these options. The iteration
proceeds until no more options remain to be selected. At this point, a
generation step produces the appropriate build files for your
configuration.

As shown in figure \ref{fig:CMakeGUI}, CMake has a different interfaces according to your system.
Refer to section~\ref{sec:compiling-linux} for Linux build instructions, \ref{sec:compiling-windows} for Windows
and \ref{sec:compiling-macosx} for Mac~OS~X.

\begin{figure}[tpb]
\centering
\includegraphics[width=0.8\textwidth]{ccmakeScreenShot.eps}
\includegraphics[width=0.8\textwidth]{CMakeSetupScreenShot.eps}
\itkcaption[Cmake user interface]{CMake interface. Top) \texttt{ccmake}, the UNIX
version based on \texttt{curses}. Bottom) \texttt{CMakeSetup}, the MS-Windows
version based on MFC.}
\label{fig:CMakeGUI}
\end{figure}

\section{Linux}
\label{sec:compiling-linux}

There are two ways of compiling OTB from sources, depending on how you want to manage dependencies.

\subsection{Building OTB and all dependencies (aka SuperBuild)}
\subsection{Building only OTB (all dependencies must be available)}

\section{Mac OS X}
\label{sec:compiling-macosx}

\section{Windows}
\label{sec:compiling-windows}

-----------

\index{CMake!OTB tree}
Running CMake initially requires that you provide three pieces of
information:
\begin{itemize}
\item Where the source code directory is located (ex.: OTB\_SOURCE\_DIR),
\item Where the object code is to be produced: (ex.: OTB\_BINARY\_DIR),
\item Where to install the binaries/libraries: (ex.: OTB\_INSTALL\_DIR).
\end{itemize}
These are referred to as the \emph{source directory}, the \emph{binary directory} and the \emph{install directory}.
We recommend setting the binary directory to be different than the source directory (an
\emph{out-of-source} build), but OTB will still build if they are set
to a directory inside the source directory (an \emph{in-source} build).


The OTB needs some external libraries to work (see table \ref{tab:installation2} below). 

\begin{center}
\begin{tiny}
\begin{table}[!htbp]
\begin{tabular}{|p{0.15\textwidth}|p{0.45\textwidth}|p{0.1\textwidth}|p{0.1\textwidth}|}
\hline
\textbf{Library} & \textbf{Web site} & \textbf{Mandatory} & \textbf{Minimum version} \\
\hline
\textbf{ITK} & \url{http://www.itk.org} & yes & 4.6.0 \\
\hline
\textbf{GDAL} & \url{http://www.gdal.org} & yes & 1.10 \\
\hline
\textbf{OSSIM} & \url{http://www.ossim.org} & yes & 1.8.6 \\
\hline
\textbf{Curl} & \url{http://www.curl.haxx.se} & no  & - \\
\hline
\textbf{FFTW} & \url{http://www.fftw.org} & no  & - \\
\hline
\textbf{libgeotiff} & \url{http://trac.osgeo.org/geotiff/} & yes & - \\
\hline
\textbf{OpenJPEG} & \url{http://code.google.com/p/openjpeg/} & no & - \\
\hline
\textbf{boost} & \url{http://www.boost.org} & yes & - \\
\hline
\textbf{openthreads} & \url{http://www.openscenegraph.org} & yes & - \\
\hline
\textbf{Mapnik} & \url{http://www.mapnik.org} & no  & - \\
\hline
\textbf{tinyXML} & \url{http://www.grinninglizard.com/tinyxml} & yes & - \\
\hline
\textbf{6S} & \url{http://6s.ltdri.org} & no & - \\
\hline
\textbf{SiftFast} & \url{http://libsift.sourceforge.net} & no  & - \\
\hline
\textbf{MuParser} & \url{http://www.muparser.sourceforge.net} & no  & - \\
\hline
\textbf{MuParserX} & \url{http://muparserx.beltoforion.de} & no  & 3.0.5 \\
\hline
\textbf{libSVM} & \url{http://www.csie.ntu.edu.tw/~cjlin/libsvm} & no  & 2.0 \\
\hline
\textbf{Qt} & \url{http://qt-project.org/} & no  & 4 \\
\hline
\textbf{OpenCV} & \url{http://opencv.org} & no  & 2 \\
\hline
\end{tabular}
\caption{Libraries used in the OTB.}
\label{tab:installation2}
\end{table}
\end{tiny}
\end{center}

To manage correctly and easily the dependencies of OTB (especially under Windows) \textbf{we strongly recommend to use the SuperBuild feature} (see next subsection \ref{sec:SuperBuild}).
Once the dependencies are built, OTB can finally be compiled : 
\begin{itemize}
	\item Just provide the location where you installed the dependencies, 
	\item Install the ones not provided by SuperBuild
    \item From the CMake interface, launch 'Configure' and 'Generate'
    \item Compile and install : call the generator used for your platform : make, nmake, jom, ... 
\end{itemize}

\textbf{\underline{Compilation awarness}}
The build process will typically take anywhere from 15 to 30 minutes depending on the performance of your system. If you decide to enable testing as part of the normal build process, about 2500 small tests programs will be compiled. This will verify that the basic components of OTB have been correctly built on your system.

Set the CMake variables \code{BUILD\_TESTING} and \code{BUILD\_EXAMPLES} to ON will activate the compilation of the examples and the tests and slow down the build process. The examples distributed with the toolbox are a helpful resource for learning how to use OTB components but are not essential for the use of the toolbox itself. The testing section includes a large number of small programs that exercise the capabilities of OTB classes. Due to the large number of tests, enabling the testing option will considerably increase the build time.  It is not desirable to enable this option for a first build of the toolbox.

The CMake variable \code{OTB\_BUILD\_DEFAULT\_MODULES} activates all usual modules. This is required to build the examples. If you want to reduce the set of enabled modules, in order to produce a smaller OTB or to avoid some third parties, you have two methods :
\begin{itemize}
  \item disable some third party modules with the CMake variables \code{OTB\_USE\_XXX}. All the modules that depend on the corresponding third party will be disabled.
  \item disable the option \code{OTB\_BUILD\_DEFAULT\_MODULES}, and enable the modules you want. The variables \code{OTBGroup\_XXX} are used to enable all modules in a group. If this variable is off, the user can select which modules in the group should be enabled, using the pattern variable \code{Module\_XXX}.
\end{itemize}

\subsection{SuperBuild}
\label{sec:SuperBuild}

The SuperBuild is a way of compiling dependencies to a project just before you build the project. Thanks to CMake and its ExternalProject module, it is possible to download a source archive, configure, compile and install it when building the main project. This feature has been used in other CMake-based projects (ITK, Slicer, ParaView,...).

In OTB, the SuperBuild is implemented with no impact on the library sources : the sources for SuperBuild are located in the 'OTB/SuperBuild' subdirectory. It is made of CMake scripts and source patches that allow to compile all the dependencies necessary for OTB. Once all the dependencies are compiled and installed, the OTB library is built using those dependencies.

The purpose is to provide an easy way to get OTB and its dependencies, whatever the platform (among Linux, MacOSX and Windows). 

\subsubsection{Requirements}
There are few requirements to use the SuperBuild :
\begin{itemize}
	\item CMake (at least 2.8.11)
	\item C and C++ compilers
	\item libtool
	\item mercurial (if you intend to download OTB sources from the repository) 
\end{itemize}

There are optional requirements depending on OTB features you want to enable :
\begin{itemize}
	\item python (for python wrapping)
    \item java (for java wrapping) 
\end{itemize}

\subsubsection{How to use it}

The SuperBuild is made like a standard CMake-type project. You should then prepare three directories (for source, build and install) :
OTB\textunderscore SOURCE\textunderscore DIR, OTB\textunderscore BINARY\textunderscore DIR, OTB\textunderscore INSTALL\textunderscore DIR.

\begin{itemize}
	\item Get OTB sources and put them in your source directory (works with revision c20055670b36 and later)
    \item Go to your build directory and configure CMake using "OTB\textunderscore SOURCE\textunderscore DIR/SuperBuild" as input
    \begin{itemize}
        \item Set the CMAKE\textunderscore INSTALL\textunderscore PREFIX to your install directory (OTB\textunderscore INSTALL\textunderscore DIR)
        \item Online mode (default) : CMake will download all the source archives required to build the dependencies in the folder pointed by the variable DOWNLOAD\textunderscore LOCATION
        \item Offline mode : if you have already the dependencies archives on your disk, set the variable DOWNLOAD\textunderscore LOCATION to the folder containing those archives. If archives names and MD5 sums match, CMake won't download them. An archive containing all the needed source archives is available on Orfeo ToolBox website, you may download it, uncompress, and point the DOWNLOAD\textunderscore LOCATION variable to the uncompressed folder (see \url{https://www.orfeo-toolbox.org/packages/SuperBuild-archives.tar.bz2}).
        \item For each dependency, you can use the variable USE\textunderscore SYSTEM\textunderscore XXX to choose whether you want this dependency to be build by SuperBuild or if a system version should be used. 
    \end{itemize}
    \item From the CMake interface, launch 'Configure' and 'Generate'
    \item Compile and install : call the generator used for your platform : make, nmake, jom, ... 
\end{itemize}

Please note that during the CMake configuration step, you may see warnings coming from the system checkup script ('SuperBuild/CMake/SystemCheckup.cmake'). They appear when CMake tries to find existing libraries on your system so it may report warnings for missing or incomplete libraries. For the SuperBuild process, you can safely ignore those warnings.

After SuperBuild compilation, your build directory will be organized with the following pattern :
\begin{itemize}
  \item BUILD\_DIRECTORY \hfill \textit{your SuperBuild build directory}
  \begin{itemize}
    \item EXPAT
    \begin{itemize}
      \item src \hfill \textit{Expat sources}
      \item build \hfill \textit{Expat build directory}
    \end{itemize}
    \item GDAL
    \begin{itemize}
      \item src \hfill \textit{Gdal sources}
      \item build \hfill \textit{Gdal build directory}
    \end{itemize}
    \item \textit{(other dependencies\ldots)}
    \item OTB
    \begin{itemize}
      \item build \hfill \textit{OTB build directory}
    \end{itemize}
  \end{itemize}
\end{itemize}

\subsubsection{Current status}

This section details the libraries that are not fully supported by SuperBuild depending on the target platform. 

\textbf{Windows MSVC 2010}
\newline
The libraries not handled by SuperBuild are : 
\begin{itemize}
	\item PCRE
\end{itemize}

\textbf{Note} : there is a limitation on the absolute path length for the source directory and build directory : it should not exceed 50 characters. In fact, this limitation is not related to the SuperBuild but comes with OTB.

\textbf{MacOSX 10.10}
\newline
The libraries not handled by SuperBuild are : 
\begin{itemize}
	\item Boost
	\item Curl
	\item PCRE
	\item PNG
	\item SWIG
	\item ZLIB
	\item QT4
\end{itemize}

\textbf{Linux}
\newline
The libraries not handled by SuperBuild are : 
\begin{itemize}
	\item Curl
	\item PCRE
	\item QT4
	\item SWIG
\end{itemize}

A wiki page is also available here : \url{http://wiki.orfeo-toolbox.org/index.php/SuperBuild}.

\subsection{Manual installation of OTB dependencies}
\label{sec:manualdependencies}

\subsubsection{Linux systems : getting a qualified Gdal library }
\label{sec:gdal}
Manual installation of OTB dependencies can easily be done under unix platforms, either directly by package managers (be sure to also install libraries headers, which are mandatory to build OTB) or by compiling from sources (follow the build instructions for each particular dependence).

However, special precautions are necessary concerning Gdal.
    
Gdal is a very rich data abstraction library, and there are many compilation time options in Gdal that will impact Gdal and therefore OTB libraries capabilities. There are also options and dependencies that could limit or prevent Orfeo ToolBox from working properly. On linux systems, there are a lot of Gdal binary packages with different configurations. It is therefore very important to know exactly your Gdal configuration before proceeding to the next step.

Here are the things to check:
\begin{itemize}
	\item Gdal version (should be greater or equal to 1.10),
	\item Gdal library should not expose any Tiff or Geotiff symbol (otherwise it may cause crashes in Orfeo ToolBox),
	\item Gdal should support BigTiff (read or write Tiff files of more than 4 Gb).
\end{itemize}

\begin{itemize}
	\item {How to check if your Gdal qualifies for Orfeo ToolBox?}
\end{itemize}
If you already have a Gdal in your system (either from the official package manager of your distribution or from your own build), here is how to check if it qualifies for Orfeo ToolBox.

Gdal version can be checked using the following command:
\begin{verbatim}
$ gdal-config --version
\end{verbatim}

If version is lower than 1.10.0, you should build or install an up-to-date Gdal library.

If you have several versions of Gdal installed in your system, be sure to run this command for the version you intend to. You can check this with:
\begin{verbatim}
$ which gdal-config
\end{verbatim}

Next step is to check if the Gdal library leaks any Tiff or Geotiff symbol. First you need to locate gdal library. If it has been installed system wide, the library will most probably be located in \texttt{/usr/lib}. Otherwise, you need to find where is \texttt{libgdal.so}.

Once you found it, run:
\begin{verbatim}
$ nm -D libgdal.so | grep XTIFFOpen
\end{verbatim}

This should show either nothing, or the following:

\begin{verbatim}
00000000004b1100 T gdal_XTIFFOpen
\end{verbatim}

If the command outputs the following:

\begin{verbatim}
00000000004b1100 T XTIFFOpen
\end{verbatim}

It means that your Gdal library leaks Tiff and Geotiff symbols. Depending on your system, this might cause random crashes in Orfeo ToolBox when reading or writing Tiff images. In this case it is advised to build or retrieve a Gdal library that does not exhibit this issue.

Last, it might be important for you to be able to read and write Tiff files of more than 4 Gb with Orfeo ToolBox. To check for BigTiff file support, choose an image on your computer and run:
\begin{verbatim}
$ gdal_translate -co "BIGTIFF=YES" my_image.png my_image.tif
\end{verbatim}

If this fails, your Gdal library or the Tiff library it is linked to does not support BigTiff files. In this case it is advised to build or retrieve a Gdal library.

\emph{Distributions and binary packages repositories known to distribute a qualified Gdal}


Among the different distributions and packages repositories that distribute a qualified Gdal, one can cite:
\begin{itemize}
\item The ubuntu-gis unstable ppa (\href{https://launchpad.net/~ubuntugis/+archive/ubuntugis-unstable}),
\item Default package for Ubuntu greater or equal to 14.04.
\end{itemize}

\emph{Building your own qualified Gdal}

If do not yet have a Gdal library or suspect that the library you have will not qualify, you can build your own library. After downloading the latest version of gdal source code, run:

\begin{verbatim}
$ export GDAL_INSTALL_DIR=$HOME/local
$ ./configure --prefix=$GDAL_INSTALL_DIR \
  --with-geotiff=internal  \
  --with-rename-internal-libtiff-symbols=yes \
  --with-rename-internal-libgeotiff-symbols=yes
$ make install
\end{verbatim}

The \texttt{GDAL\_INSTALL\_DIR} environment variable allows to choose where you will install your Gdal library. It is NOT advised to install it system wide. It is recommended to choose a dedicated directory in your home folder. The configuration options will ensure that Gdal will build internal Tiff and Geotiff libraries that supports BigTiff files, while renaming their symbols so that they will not be leaked outside of the library. Depending on your need, feel free to enable more Gdal configuration options.

Once this is done, be sure to update your environment variables:
\begin{verbatim}
$ export PATH=$PATH:$GDAL_INSTALL_DIR/bin
$ export LD_LIBRARY_PATH=$LD_LIBRARY_PATH:$GDAL_INSTALL_DIR/lib
$ export GDAL_DATA=$GDAL_INSTALL_DIR/share/gdal
\end{verbatim}

Also note that with this Gdal configuration, you will still need to have Tiff and Geotiff libraries and headers available on your system to build OTB (due to third parties requiring them). However, you can use the system packages for those.

\subsection{Custom OTB}
Since OTB is modularized, it is possible to only build some modules instead of the whole set. 
You can deactivate each module (and the ones that depend on it) one by one from the CMake interface. 
You just have to switch off the CMake variable OTB\_BUILD\_DEFAULT\_MODULES, press Configure, and then switch off each Module\_module\_name variable (\*).
To provide an overview on how things work, the option COMPONENTS of the CMake command find\_package is used in order to only load the requested modules.
This module-specific list prevent CMake from performing a blind search; it is also a convienent way to monitor the dependencies of each module.

\begin{verbatim}
find_package(OTB COMPONENTS OTBCommon OTBTransform [...])
\end{verbatim} 

Some of the OTB capabilities are considered as optional, and you can deactivate the related modules thanks to a set of CMake variables starting with OTB\_USE\_\ldots 

\begin{center}
\begin{tiny}
\begin{table}[!htbp]
\begin{tabular}{|l|l|p{0.52\textwidth}|}
\hline
\textbf{CMake variable} & \textbf{3rd party module} & \textbf{Modules depending on it} \\
\hline
\textbf{OTB\_USE\_LIBKML} & OTBlibkml & OTBKMZWriter OTBIOKML OTBAppKMZ \\
\hline
\textbf{OTB\_USE\_QT4} & OTBQt4 & OTBQtWidget \\
\hline
\textbf{OTB\_USE\_OPENCV} & OTBOpenCV & \\
\hline
\textbf{OTB\_USE\_MUPARSERX} & OTBMuParserX & OTBMathParserX OTBAppMathParserX \\
\hline
\textbf{OTB\_USE\_OPENJPEG} & OTBOpenJPEG & OTBIOJPEG2000 \\
\hline
\textbf{OTB\_USE\_CURL} & OTBCurl & \\
\hline
\textbf{OTB\_USE\_MUPARSER} & OTBMuParser & OTBMathParser OTBDempsterShafer OTBAppClassification OTBAppMathParser OTBAppStereo OTBAppProjection OTBAppSegmentation OTBAppClassification OTBRoadExtraction OTBRCC8 OTBCCOBIA OTBAppSegmentation OTBMeanShift OTBAppSegmentation OTBMeanShift OTBAppSegmentation \\
\hline
\textbf{OTB\_USE\_LIBSVM} & OTBLibSVM & OTBSVMLearning \\
\hline
\textbf{OTB\_USE\_MAPNIK} & OTBMapnik & OTBVectorDataRendering \\
\hline
\textbf{OTB\_USE\_6S} & OTB6S & OTBOpticalCalibration OTBAppOpticalCalibration OTBSimulation \\
\hline
\textbf{OTB\_USE\_SIFTFAST} & OTBSiftFast & \\
\hline
\end{tabular}
\caption{Third parties and related modules.}
\label{tab:optional}
\end{table}
\end{tiny}
\end{center}

The table above shows which third party modules are associated to these variables, and which other modules depend on it. \textbf{It is very important to notice that these variable override the variable OTB\_BUILD\_DEFAULT\_MODULES.}.

Every CMake variable mentionned in this subsection is set to OFF by default except OTB\_USE\_6S and OTB\_USE\_SIFTFAST (which are internal dependencies) so that first build will always be successful. The list of default modules is then the complete module list minus the ones disabled by third parties : 

\begin{multicols}{3}
\small
OTB6S \\
OTBAnomalyDetection \\
OTBAppChangeDetection \\
OTBAppDescriptors \\
OTBAppDimensionalityReduction \\
OTBAppEdge \\
OTBAppFiltering \\
OTBAppFusion \\
OTBAppHyperspectral \\
OTBAppImageUtils \\
OTBAppIndices \\
OTBApplicationEngine \\
OTBAppMoments \\
OTBAppMorphology \\
OTBAppOpticalCalibration \\
OTBAppSARCalibration \\
OTBAppTest \\
OTBAppTextures \\
OTBAppVectorDataTranslation \\
OTBAppVectorUtils \\
OTBBoost \\
OTBBoostAdapters \\
OTBCarto \\
OTBChangeDetection \\
OTBCloudDetection \\
OTBColorMap \\
OTBCommandLine \\
OTBCommandLineParser \\
OTBCommon \\
OTBComplexImage \\
OTBConversion \\
OTBConvolution \\
OTBCorner \\
OTBCurlAdapters \\
OTBDEM \\
OTBDensity \\
OTBDescriptors \\
OTBDimensionalityReduction \\
OTBDisparityMap \\
OTBDisplacementField \\
OTBEdge \\
OTBEndmembersExtraction \\
OTBExtendedFilename \\
OTBFuzzy \\
OTBGDAL \\
OTBGdalAdapters \\
OTBGeoTIFF \\
OTBImageBase \\
OTBImageIO \\
OTBImageManipulation \\
OTBImageNoise \\
OTBIndices \\
OTBInterpolation \\
OTBIOBSQ \\
OTBIOGDAL \\
OTBIOLUM \\
OTBIOMSTAR \\
OTBIOMW \\
OTBIOONERA \\
OTBIORAD \\
OTBIOTileMap \\
OTBIOXML \\
OTBITK \\
OTBLabelling \\
OTBLabelMap \\
OTBLandSatClassifier \\
OTBLearningBase \\
OTBMajorityVoting \\
OTBMarkov \\
OTBMetadata \\
OTBMetrics \\
OTBMoments \\
OTBMorphologicalProfiles \\
OTBMorphologicalPyramid \\
OTBObjectDetection \\
OTBObjectList \\
OTBOGRProcessing \\
OTBOpenThreads \\
OTBOpenThreadsAdapters \\
OTBOpticalCalibration \\
OTBOssim \\
OTBOSSIMAdapters \\
OTBOssimPlugins \\
OTBPanSharpening \\
OTBPath \\
OTBPointSet \\
OTBPolarimetry \\
OTBProjection \\
OTBSARCalibration \\
OTBSeamCarving \\
OTBSiftFast \\
OTBSimulation \\
OTBSmoothing \\
OTBSOM \\
OTBSpatialObjects \\
OTBStatistics \\
OTBStereo \\
OTBStreaming \\
OTBSupervised \\
OTBSWIGWrapper \\
OTBTestKernel \\
OTBTextures \\
OTBTimeSeries \\
OTBTinyXML \\
OTBTransform \\
OTBUnmixing \\
OTBUrbanArea \\
OTBVectorDataBase \\
OTBVectorDataIO \\
OTBVectorDataManipulation \\
OTBWatersheds \\
OTBWavelet \\
\end{multicols}


\section{Known issues}
\label{sec:knownissues}

\begin{itemize}
\item  openjpeg/ITK 
\end{itemize}

It is important to know that the OpenJpeg library doesn't support name mangling since version 2.0. 
As a consequence, if other libraries linked by your project already contain OpenJpeg, there may be a symbol conflict at run-time. 
For instance, this was observed with OTB build on a recent ITK version (ver. 4). 
The ITK library already had a version of OpenJpeg in libitkopenjpeg-*.so, which contained the OpenJpeg symbols un-wrapped.
These symbols were also loaded by the GDAL driver but only the first ones were used, which caused a crash. 

Hopefully, thanks to the modular architecture of ITK, the library libitkopenjpeg-*.so is not imported anymore inside OTB.
However the OpenJPEG headers may be present in ITK include directory. As the current architecture doesn't allow to tune 
include order between modules, the OpenJPEG header from ITK can be included before your own OpenJPEG install. There are
two ways to avoid this situation :
\begin{itemize}
\item Use an ITK without GDCM nor ITKReview (only these modules depend on OpenJPEG)
\item Hide the header openjpeg.h in the ITK include directory.
\end{itemize}

More information can be found here : \url{http://wiki.orfeo-toolbox.org/index.php/JPEG2000_with_GDAL_OpenJpeg_plugin}

\begin{itemize}
\item  libkml / Ubuntu 12.04 
\end{itemize}

Another issue is related to the official package of libkml under Ubuntu 12.4.
Until this problem is addressed, users of this plateform should disable the option OTB\_USE\_KML, so that OTB won't be built with this third-party.

