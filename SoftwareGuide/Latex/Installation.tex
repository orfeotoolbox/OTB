\setcounter{secnumdepth}{3}

\chapter{Compiling OTB from source}
\label{chapter:Installation}
\index{Installation}

There are two ways to install OTB library on your system: installing from a binary distribution or compiling from sources. 
You can find information about the installation of binary packages for OTB-Applications and Monteverdi and Monteverdi2 (if they are available on your platform) in the OTB-Cookbook.
This chapter covers installation from sources, also known as compiling.

OTB has been developed and tested across different combinations of operating systems, compilers, and hardware platforms including MS-Windows, Linux on Intel-compatible hardware and Mac OSX.  It is known to work with the following compilers in 32/64 bit:
\begin{itemize}
\item Visual Studio 2010 and higher compiler on MS-Windows
\item GCC 4.1 and higher on Unix/Linux systems
\item Clang on MacOSX (10.8 and higher) systems
\end{itemize}

\index{CMake}
The challenge of supporting OTB across platforms has been solved through the use of CMake, a cross-platform, open-source
build system. CMake is used to control the software compilation process using simple platform and compiler independent
configuration files.  CMake generates native makefiles and workspaces that can be used in the compiler environment of
your choice. CMake is quite sophisticated: it supports complex environments requiring system configuration, compiler
feature testing, and code generation.

CMake generates Makefiles under UNIX systems and generates Visual Studio workspaces under Windows (and appropriate build
files for other compilers like Borland). The information used by CMake is provided by \code{CMakeLists.txt} files that
are present in every directory of the OTB source tree. These files contain information that the user provides to CMake
at configuration time. Typical information includes paths to utilities in the system and the selection of software
options specified by the user.

CMake runs in an interactive mode in that you iteratively select
options and configure according to these options. The iteration
proceeds until no more options remain to be selected. At this point, a
generation step produces the appropriate build files for your
configuration.

As shown in figure \ref{fig:CMakeGUI}, CMake has a different interfaces according to your system.
Refer to section~\ref{sec:compiling-linux} for Linux build instructions, 
\ref{sec:compiling-macosx} for Mac~OS~X,
and \ref{sec:compiling-windows} for Windows.

\begin{figure}[tpb]
\centering
\includegraphics[width=0.8\textwidth]{ccmakeScreenShot.eps}
\includegraphics[width=0.8\textwidth]{CMakeSetupScreenShot.eps}
\itkcaption[Cmake user interface]{CMake interface. Top) \texttt{ccmake}, the UNIX
version based on \texttt{curses}. Bottom) \texttt{CMakeSetup}, the MS-Windows
version based on MFC.}
\label{fig:CMakeGUI}
\end{figure}

OTB depends on a number of external libraries.
Some are mandatory, meaning that OTB cannot be compiled without them, while others are optional and can be activated or
not during the build process.
See table \ref{tab:otb-dependencies} for the full list of dependencies.
\begin{center}
\begin{tiny}
\begin{table}[!htbp]
\begin{tabular}{|p{0.15\textwidth}|p{0.45\textwidth}|p{0.1\textwidth}|p{0.1\textwidth}|}
\hline
\textbf{Library} & \textbf{Web site} & \textbf{Mandatory} & \textbf{Minimum version} \\
\hline
\textbf{ITK} & \url{http://www.itk.org} & yes & 4.6.0 \\
\hline
\textbf{GDAL} & \url{http://www.gdal.org} & yes & 1.10 \\
\hline
\textbf{OSSIM} & \url{http://www.ossim.org} & yes & 1.8.6 \\
\hline
\textbf{Curl} & \url{http://www.curl.haxx.se} & no  & - \\
\hline
\textbf{FFTW} & \url{http://www.fftw.org} & no  & - \\
\hline
\textbf{libgeotiff} & \url{http://trac.osgeo.org/geotiff/} & yes & - \\
\hline
\textbf{OpenJPEG} & \url{http://code.google.com/p/openjpeg/} & no & - \\
\hline
\textbf{boost} & \url{http://www.boost.org} & yes & - \\
\hline
\textbf{openthreads} & \url{http://www.openscenegraph.org} & yes & - \\
\hline
\textbf{Mapnik} & \url{http://www.mapnik.org} & no  & - \\
\hline
\textbf{tinyXML} & \url{http://www.grinninglizard.com/tinyxml} & yes & - \\
\hline
\textbf{6S} & \url{http://6s.ltdri.org} & no & - \\
\hline
\textbf{SiftFast} & \url{http://libsift.sourceforge.net} & no  & - \\
\hline
\textbf{MuParser} & \url{http://www.muparser.sourceforge.net} & no  & - \\
\hline
\textbf{MuParserX} & \url{http://muparserx.beltoforion.de} & no  & 3.0.5 \\
\hline
\textbf{libSVM} & \url{http://www.csie.ntu.edu.tw/~cjlin/libsvm} & no  & 2.0 \\
\hline
\textbf{Qt} & \url{http://qt-project.org/} & no  & 4 \\
\hline
\textbf{OpenCV} & \url{http://opencv.org} & no  & 2 \\
\hline
\end{tabular}
\caption{External libraries used in OTB.}
\label{tab:otb-dependencies}
\end{table}
\end{tiny}
\end{center}

\section{Linux}
\label{sec:compiling-linux}

\subsection{Setting up the build environment}

The first thing to do is to create a directory for working with OTB.
This guide will use \texttt{$\sim$/OTB} but you are free to choose something else.
In this directory, there will be three locations:
\begin{itemize}
\item \texttt{$\sim$/OTB/otb} for the source file obtained from the git repository
\item \texttt{$\sim$/OTB/build} for the intermediate build objects, CMake specific files, libraries and binaries.
\item \texttt{$\sim$/OTB/install}, the installation directory for OTB once it is built.
A system location (\texttt{/usr/local} for example) can also be used, but installing locally is more flexible and does
not require root access.
\end{itemize}
To setup this structure, the following commands can be used:
\begin{verbatim}
$ mkdir ~/OTB
$ cd ~/OTB
$ git clone https://git@git.orfeo-toolbox.org/git/otb.git
$ mkdir build
$ mkdir install
\end{verbatim}

The OTB project uses a git branching model where \texttt{develop} is the current development version.
It contains the latest patches and represents the work in progress towards the next release.
For more information on OTB and git, including how to decide which branch to want to compile, please see the
OTB wiki page at \url{http://wiki.orfeo-toolbox.org/index.php/Git}.

Checkout the relevant branch now:
\begin{verbatim}
$ cd ~/OTB/otb
$ git checkout develop
\end{verbatim}

Now you must decide which build method you will use.
There are two ways of compiling OTB from sources, depending on how you want to manage dependencies.
Both methods rely on CMake.
\begin{itemize}
\item SuperBuild (go to section~\ref{sec:installation-linux-superbuild}). All OTB dependencies are automatically downloaded and compiled.
This method is the easiest to use and provides a complete OTB with minimal effort.
\item Normal build (go to section~\ref{sec:installation-linux-normalbuild}). OTB dependencies must already be compiled and available on your system.
This method requires more work but provides more flexibility.
\end{itemize}
If you do not know which method to use and just want to compile OTB with all its modules, use SuperBuild.

\begin{center}
\begin{tiny}
\begin{table}[!htbp]
\begin{tabular}{p{0.35\textwidth}p{0.65\textwidth}}
\hline
\textbf{CMake variable} & \textbf{Value} \\
\hline
\texttt{CMAKE\_INSTALL\_PREFIX}         & Installation directory, target for \texttt{make install} \\
\texttt{BUILD\_EXAMPLES}                & Activate compilation of OTB examples \\
\texttt{BUILD\_TESTING}                 & Activate compilation of the tests \\
\texttt{OTB\_BUILD\_DEFAULT\_MODULES}   & Activate all usual modules, required to build the examples \\
\texttt{OTB\_USE\_\textit{XXX}}         & Activate module \textit{XXX} \\
\texttt{OTBGroup\_\textit{XXX}}         & Enable modules in the group \textit{XXX} \\
\texttt{OTB\_DATA\_ROOT}                & otb-data repository \\
\texttt{OTB\_WRAP\_PYTHON}              & Enable Python wrapper \\
\texttt{OTB\_WRAP\_JAVA}                & Enable Java wrapper \\

\hline
\multicolumn{2}{l}{\small \textbf{SuperBuild only}} \\ 
\texttt{DOWNLOAD\_LOCATION}             & Location to download dependencies \\
\texttt{USE\_SYSTEM\_\textit{XXX}}      & Use the system's \textit{XXX} library \\

\hline
\end{tabular}
\caption{Important CMake configuration variables in OTB}
\label{tab:installation-cmake-variables}
\end{table}
\end{tiny}
\end{center}

\subsection{Building OTB and all dependencies (aka SuperBuild)}
\label{sec:installation-linux-superbuild}

The SuperBuild is a way of compiling dependencies to a project just before you build the project. Thanks to CMake and
its ExternalProject module, it is possible to download a source archive, configure, compile and install it when building
the main project. This feature has been used in other CMake-based projects (ITK, Slicer, ParaView,...).
In OTB, the SuperBuild is implemented with no impact on the library sources : the sources for SuperBuild are located in
the 'OTB/SuperBuild' subdirectory. It is made of CMake scripts and source patches that allow to compile all the
dependencies necessary for OTB. Once all the dependencies are compiled and installed, the OTB library is built using
those dependencies.

OTB's compilation is customized by specifying configuration variables.
The most important configuration variables are shown in table~\ref{tab:installation-cmake-variables}.
The simplest way to provide configuration variables is via the command line \texttt{-D} option:
\begin{verbatim}
$ cd ~/OTB/build
$ cmake -D CMAKE_INSTALL_LOCATION=~/OTB/install ../otb/SuperBuild
\end{verbatim}
A pre-load script can also be used with the \texttt{-C} options (see
\url{https://cmake.org/cmake/help/v3.4/manual/cmake.1.html#options}).
Another option is to set variables manually with \texttt{cmake-gui} or \texttt{ccmake}.

During the configuration step, the SuperBuild will detect any existing dependencies installed as systems libraries.
Wheter to use them can be controlled via the \texttt{USE\_SYSTEM\_\textit{XXX}} (see
table~\ref{tab:installation-cmake-variables}).

SuperBuild downloads dependencies into the \texttt{DOWNLOAD\_LOCATION} directory, which will be
\texttt{$\sim$/OTB/build/Downloads} in our example.
Dependencies can be downloaded manually into this directory before the compilation step.
This can be usefull if you wish to bypass a proxy, intend to compile OTB without an internet conection, or other network
constraint.

You are now ready to compile OTB!
Simply use the make command (other targets can be generated with CMake's \texttt{-G} option):
\begin{verbatim}
$ cd ~/OTB/build
$ make
\end{verbatim}

The installation target will copy the binaries and libraries to the installation location:
\begin{verbatim}
$ make install
\end{verbatim}

A wiki page detailing the status of SuperBuild on various platforms is also available here:
\url{http://wiki.orfeo-toolbox.org/index.php/SuperBuild}.

\subsection{Building only OTB (all dependencies must be available)}
\label{sec:installation-linux-normalbuild}

Once all OTB dependencies are availables on your system, use CMake to generate a Makefile:
\begin{verbatim}
$ cd ~/OTB/build
$ cmake -C configuration.cmake ../otb
\end{verbatim}
The script \texttt{configuration.cmake} needs to contain dependencies location if CMake cannot find them automatically.
This can be done with the \texttt{\textit{XXX}\_DIR} variables containing the directories which contain the
FindXXX.cmake scripts, or with the \texttt{\textit{XXX}\_INCLUDEDIR} and \texttt{\textit{XXX}\_LIBRARY} variables.

Additionally, decide which module you wish to enable, together with tests and examples.
Refer to table~\ref{tab:installation-cmake-variables} for the list of CMake variables.

Since OTB is modularized, it is possible to only build some modules instead of the whole set. 
To deactivate a module (and the ones that depend on it) switch off the CMake variable OTB\_BUILD\_DEFAULT\_MODULES,
configure, and then switch off each \texttt{Module\_module\_name} variable.
To provide an overview on how things work, the option \texttt{COMPONENTS} of the CMake command find\_package is used in
order to only load the requested modules.
This module-specific list prevent CMake from performing a blind search; it is also a convienent way to monitor the
dependencies of each module.
\begin{verbatim}
find_package(OTB COMPONENTS OTBCommon OTBTransform [...])
\end{verbatim} 

Some of the OTB capabilities are considered as optional, and you can deactivate the related modules thanks to a set of
CMake variables starting with \texttt{OTB\_USE\_\textit{XXX}}.
Table~\ref{tab:optional} shows which modules are associated to these variables. It is very important to notice that
these variable override the variable OTB\_BUILD\_DEFAULT\_MODULES.

You are now ready to compile OTB!
Simply use the make command (other targets can be generated with CMake's \texttt{-G} option):
\begin{verbatim}
$ make
\end{verbatim}

The installation target will copy the binaries and libraries to the installation location:
\begin{verbatim}
$ make install
\end{verbatim}

\begin{center}
\begin{tiny}
\begin{table}[!htbp]
\begin{tabular}{|l|l|p{0.52\textwidth}|}
\hline
\textbf{CMake variable} & \textbf{3rd party module} & \textbf{Modules depending on it} \\
\hline
\textbf{OTB\_USE\_LIBKML} & OTBlibkml & OTBKMZWriter OTBIOKML OTBAppKMZ \\
\hline
\textbf{OTB\_USE\_QT4} & OTBQt4 & OTBQtWidget \\
\hline
\textbf{OTB\_USE\_OPENCV} & OTBOpenCV & \\
\hline
\textbf{OTB\_USE\_MUPARSERX} & OTBMuParserX & OTBMathParserX OTBAppMathParserX \\
\hline
\textbf{OTB\_USE\_OPENJPEG} & OTBOpenJPEG & OTBIOJPEG2000 \\
\hline
\textbf{OTB\_USE\_CURL} & OTBCurl & \\
\hline
\textbf{OTB\_USE\_MUPARSER} & OTBMuParser & OTBMathParser OTBDempsterShafer OTBAppClassification OTBAppMathParser OTBAppStereo OTBAppProjection OTBAppSegmentation OTBAppClassification OTBRoadExtraction OTBRCC8 OTBCCOBIA OTBAppSegmentation OTBMeanShift OTBAppSegmentation OTBMeanShift OTBAppSegmentation \\
\hline
\textbf{OTB\_USE\_LIBSVM} & OTBLibSVM & OTBSVMLearning \\
\hline
\textbf{OTB\_USE\_MAPNIK} & OTBMapnik & OTBVectorDataRendering \\
\hline
\textbf{OTB\_USE\_6S} & OTB6S & OTBOpticalCalibration OTBAppOpticalCalibration OTBSimulation \\
\hline
\textbf{OTB\_USE\_SIFTFAST} & OTBSiftFast & \\
\hline
\end{tabular}
\caption{Third parties and related modules.}
\label{tab:optional}
\end{table}
\end{tiny}
\end{center}

\section{Mac OS X}
\label{sec:compiling-macosx}

\section{Windows}
\label{sec:compiling-windows}

\section{Linux systems : getting a qualified Gdal library }
\label{sec:gdal}
Manual installation of OTB dependencies can easily be done under unix platforms, either directly by package managers (be sure to also install libraries headers, which are mandatory to build OTB) or by compiling from sources (follow the build instructions for each particular dependence).

However, special precautions are necessary concerning Gdal.
    
Gdal is a very rich data abstraction library, and there are many compilation time options in Gdal that will impact Gdal and therefore OTB libraries capabilities. There are also options and dependencies that could limit or prevent Orfeo ToolBox from working properly. On linux systems, there are a lot of Gdal binary packages with different configurations. It is therefore very important to know exactly your Gdal configuration before proceeding to the next step.

Here are the things to check:
\begin{itemize}
	\item Gdal version (should be greater or equal to 1.10),
	\item Gdal library should not expose any Tiff or Geotiff symbol (otherwise it may cause crashes in Orfeo ToolBox),
	\item Gdal should support BigTiff (read or write Tiff files of more than 4 Gb).
\end{itemize}

\begin{itemize}
	\item {How to check if your Gdal qualifies for Orfeo ToolBox?}
\end{itemize}
If you already have a Gdal in your system (either from the official package manager of your distribution or from your own build), here is how to check if it qualifies for Orfeo ToolBox.

Gdal version can be checked using the following command:
\begin{verbatim}
$ gdal-config --version
\end{verbatim}

If version is lower than 1.10.0, you should build or install an up-to-date Gdal library.

If you have several versions of Gdal installed in your system, be sure to run this command for the version you intend to. You can check this with:
\begin{verbatim}
$ which gdal-config
\end{verbatim}

Next step is to check if the Gdal library leaks any Tiff or Geotiff symbol. First you need to locate gdal library. If it has been installed system wide, the library will most probably be located in \texttt{/usr/lib}. Otherwise, you need to find where is \texttt{libgdal.so}.

Once you found it, run:
\begin{verbatim}
$ nm -D libgdal.so | grep XTIFFOpen
\end{verbatim}

This should show either nothing, or the following:

\begin{verbatim}
00000000004b1100 T gdal_XTIFFOpen
\end{verbatim}

If the command outputs the following:

\begin{verbatim}
00000000004b1100 T XTIFFOpen
\end{verbatim}

It means that your Gdal library leaks Tiff and Geotiff symbols. Depending on your system, this might cause random crashes in Orfeo ToolBox when reading or writing Tiff images. In this case it is advised to build or retrieve a Gdal library that does not exhibit this issue.

Last, it might be important for you to be able to read and write Tiff files of more than 4 Gb with Orfeo ToolBox. To check for BigTiff file support, choose an image on your computer and run:
\begin{verbatim}
$ gdal_translate -co "BIGTIFF=YES" my_image.png my_image.tif
\end{verbatim}

If this fails, your Gdal library or the Tiff library it is linked to does not support BigTiff files. In this case it is advised to build or retrieve a Gdal library.

\emph{Distributions and binary packages repositories known to distribute a qualified Gdal}


Among the different distributions and packages repositories that distribute a qualified Gdal, one can cite:
\begin{itemize}
\item The ubuntu-gis unstable ppa (\href{https://launchpad.net/~ubuntugis/+archive/ubuntugis-unstable}),
\item Default package for Ubuntu greater or equal to 14.04.
\end{itemize}

\emph{Building your own qualified Gdal}

If do not yet have a Gdal library or suspect that the library you have will not qualify, you can build your own library. After downloading the latest version of gdal source code, run:

\begin{verbatim}
$ export GDAL_INSTALL_DIR=$HOME/local
$ ./configure --prefix=$GDAL_INSTALL_DIR \
  --with-geotiff=internal  \
  --with-rename-internal-libtiff-symbols=yes \
  --with-rename-internal-libgeotiff-symbols=yes
$ make install
\end{verbatim}

The \texttt{GDAL\_INSTALL\_DIR} environment variable allows to choose where you will install your Gdal library. It is NOT advised to install it system wide. It is recommended to choose a dedicated directory in your home folder. The configuration options will ensure that Gdal will build internal Tiff and Geotiff libraries that supports BigTiff files, while renaming their symbols so that they will not be leaked outside of the library. Depending on your need, feel free to enable more Gdal configuration options.

Once this is done, be sure to update your environment variables:
\begin{verbatim}
$ export PATH=$PATH:$GDAL_INSTALL_DIR/bin
$ export LD_LIBRARY_PATH=$LD_LIBRARY_PATH:$GDAL_INSTALL_DIR/lib
$ export GDAL_DATA=$GDAL_INSTALL_DIR/share/gdal
\end{verbatim}

Also note that with this Gdal configuration, you will still need to have Tiff and Geotiff libraries and headers available on your system to build OTB (due to third parties requiring them). However, you can use the system packages for those.

\section{Known issues}
\label{sec:knownissues}

\begin{itemize}
\item  openjpeg/ITK 
\end{itemize}

It is important to know that the OpenJpeg library doesn't support name mangling since version 2.0. 
As a consequence, if other libraries linked by your project already contain OpenJpeg, there may be a symbol conflict at run-time. 
For instance, this was observed with OTB build on a recent ITK version (ver. 4). 
The ITK library already had a version of OpenJpeg in libitkopenjpeg-*.so, which contained the OpenJpeg symbols un-wrapped.
These symbols were also loaded by the GDAL driver but only the first ones were used, which caused a crash. 

Hopefully, thanks to the modular architecture of ITK, the library libitkopenjpeg-*.so is not imported anymore inside OTB.
However the OpenJPEG headers may be present in ITK include directory. As the current architecture doesn't allow to tune 
include order between modules, the OpenJPEG header from ITK can be included before your own OpenJPEG install. There are
two ways to avoid this situation :
\begin{itemize}
\item Use an ITK without GDCM nor ITKReview (only these modules depend on OpenJPEG)
\item Hide the header openjpeg.h in the ITK include directory.
\end{itemize}

More information can be found here : \url{http://wiki.orfeo-toolbox.org/index.php/JPEG2000_with_GDAL_OpenJpeg_plugin}

\begin{itemize}
\item  libkml / Ubuntu 12.04 
\end{itemize}

Another issue is related to the official package of libkml under Ubuntu 12.4.
Until this problem is addressed, users of this plateform should disable the option OTB\_USE\_KML, so that OTB won't be built with this third-party.

